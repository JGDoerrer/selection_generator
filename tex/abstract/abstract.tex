\documentclass[12pt]{article}

% Packages
\usepackage[utf8]{inputenc} % Input encoding
\usepackage{amsmath} % Mathematical symbols and environments
\usepackage{graphicx} % Include graphics
\usepackage{hyperref} % Hyperlinks
\usepackage{geometry} % Page layout
\usepackage{setspace} % Line spacing

% Page layout
\geometry{a4paper, margin=1in, top=0.5in}

% Document starts
\begin{document}

% Title
\title{Finding Lower Bounds for the Number of Comparisons in Selection Algorithms}
\author{Julius von Smercek, Josua Dörrer, Konrad Gendle, \\ Andreas Steding, Johanna Hofmann}
\date{June 26, 2024}
\maketitle

\section*{Abstract}
This research project aims to find worst case optimal comparison algorithms for selecting the i-th smallest of n elements of
a set for n up to 15 with computer search.
Explicitly, we apply computer search to determine the optimal worst-case number of comparisons for selecting a single element from a set of initially unordered elements.
Our method comprises the three main approaches forward search, backward search and bidirectional search. 
Additionally, we harness the notion of compatible solutions in all of the three searches.
For the forward search approach besides pruning, poset canonification and heuristics, multithreading was applied successfully.
In backward search the greatest challenge is posed by the task of calculating the predecessors. 
This was tackled by a four-step procedure that can be performed within reasonable time and storage costs.

Time measurements showed that the backward search performed slightly faster for  smaller $n$ and $i$. For higher values, up to $n=15$ and $i=7$ the backward search consumed only about $24\%$ of the time taken by the forward search.
Overall, we could not only confirm solutions of previous research but also improve the lower bound for $n=15$ and $i=4$ by one comparison.
Furthermore, we can disprove the conjecture that “pair-forming algorithms”, in which the first comparison of any singleton with another singleton is performed, do not lead to suboptimal results. 
For $n = 12$ and $i = 4$, this assumption
gives a bound of 20 comparisons, which does not correspond
to the optimal bound of 19 comparisons.

\end{document}