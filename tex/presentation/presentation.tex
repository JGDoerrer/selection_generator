\documentclass{beamer}
\usetheme{Boadilla}

\beamertemplatenavigationsymbolsempty 

\usepackage{diagbox}

\title{Selecting the `i'th largest of `n' elements}
\subtitle{with the fewest possible comparisons.}
\author{Die Bildungselite}
% \institute{FMI - University of Stuttgart}
\date{\today}


\begin{document}
\begin{frame}
    \titlepage
\end{frame}

\section{Poset}

\begin{frame}
    \frametitle{\insertsection}

    Was ist ein Poset?

    ...

\end{frame}

\section{Forward Search}

\begin{frame}
    \frametitle{\insertsection}

    Forward search

\end{frame}

\subsection{Multithreading}

\begin{frame}
    \frametitle{\insertsubsection}



\end{frame}

\section{Backward Search}

\begin{frame}
    \frametitle{\insertsection}

    Backward search

\end{frame}

\section{Bidirectional Search}

\begin{frame}
    \frametitle{\insertsection}

    Bidirectional search

\end{frame}

\section{Results}

\begin{frame}
    \frametitle{\insertsection}

    \begin{table}
        \centering
        \begin{tabular}{c|cccccccc}
            \backslashbox{$n$}{$i$} & 0  & 1  & 2  & 3  & 4  & 5  & 6  & 7  \\ \hline
            1                       & 0                                     \\
            2                       & 1                                     \\
            3                       & 2  & 3                                \\
            4                       & 3  & 4                                \\
            5                       & 4  & 6  & 6                           \\
            6                       & 5  & 7  & 8                           \\
            7                       & 6  & 8  & 10 & 10                     \\
            8                       & 7  & 9  & 11 & 12                     \\
            9                       & 8  & 11 & 12 & 14 & 14                \\
            10                      & 9  & 12 & 14 & 15 & 16                \\
            11                      & 10 & 13 & 15 & 17 & 18 & 18           \\
            12                      & 11 & 14 & 17 & 18 & 19 & 20           \\
            13                      & 12 & 15 & 18 & 20 & 21 & 22 & 23      \\
            14                      & 13 & 16 & 19 & 21 & 23 & 24 & 25      \\
            15                      & 14 & 17 & 20 & 23 & 24 & 26 & 26 & 27 \\
        \end{tabular}
    \end{table}

\end{frame}

\end{document}